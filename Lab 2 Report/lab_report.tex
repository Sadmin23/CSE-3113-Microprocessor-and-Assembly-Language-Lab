\documentclass[11pt]{article}

\usepackage{graphicx}
\usepackage{url}

\begin{document}

\begin{titlepage}
	\begin{center}
    	\includegraphics[scale=0.10]{du.png}\par
		\begin{Huge}
			\textsc{University of Dhaka}\par
		\end{Huge}
		\begin{Large}
			Department of Computer Science and Engineering\par \vspace{1cm}
			CSE-3111 : Microprocessor and Assembly Language Lab \\[12pt]	
			Lab 2 Report
		\end{Large}
	\end{center}  	
	\begin{large}
		\textbf{Submitted By:\\[12pt]}
			Name: Md Sadmin Tahmid Khan\\[8pt]
			Roll No : 35\\[12pt]
		\textbf{Submitted On : \\[12pt]}
			January 31, 2023\\[20pt]
		\textbf{Submitted To :\\[12pt]}
			Dr. Upama Kabir\\[12pt]
                Dr. Md. Mustafizur Rahman\\[12pt]
	\end{large}
\end{titlepage}

\section{Introduction}
Assembly language is a low-level programming language used to program computers and other devices. It is used to create programs that can operate on a particular microprocessor or microcontroller and is a symbolic representation of machine instructions. An assembler converts assembly language commands into machine code, and the computer then executes the generated machine code.

\section{Objectives}
The objectives of this lab experiment.
\begin{itemize}
    \item Demonstrate proficiency in using KEIL μVision 5 IDE

    \item Understand register based assembly programming of basic arithmetic operation. 

    \item Implement simple arithmetic operations including Addition, Subtraction and Multiplication using.

    \item Understand the use of constants, load and arithmetic operations in a simple assembly language.
\end{itemize}
%%%%
%%%%
\section{Theory}
In assembly language, arithmetic operations are performed using register-based operations. Registers are small storage units inside the central processing unit (CPU) that hold data that is being processed. The following are some basic register-based arithmetic operations:
\begin{itemize}
    \item Addition: The ADD instruction is used to add two values stored in two different registers. The result is stored in another register.

    \item Subtraction: The SUB instruction is used to subtract two values stored in two different registers. The result is stored in another register.

    \item Multiplication: The MUL instruction is used to multiply two values stored in two different registers. The result is stored in another register.
\end{itemize}
\section{Conclusion}
In all task of the experiments we had to store the answer in a particular register. The values stored in the registers are kept in hexadecimal values. I was able to complete all tasks and show the output. One problem I faced was to assign constant values as a intermediate step; this problem was solved after exploring few websites.
\section{Codes}
\href{https://github.com/Sadmin23/CSE-3113-Microprocessor-and-Assembly-Language-Lab}

\end{document}